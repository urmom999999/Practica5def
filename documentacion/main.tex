\documentclass{article}
\usepackage[utf8]{inputenc}
\usepackage{graphicx,float}
\usepackage{hyperref}
\usepackage{listings}
\usepackage{verbatim}

\graphicspath{{imagenes/}}

\title{\textbf{Programación de Servicios y Procesos: Práctica de Docker Local}}
\author{Álvaro Del Valle Fernández}


\begin{document}
\maketitle
\begin{figure}[H]
    \centering
    \includegraphics[width=4in]{odoo.png}
\end{figure}
\newpage
\section{Introducción}
En esta práctica completaré 4 ejercicios diferentes en Odoo usando como referencia los archivos del repositorio de git entregado.\\
Antes de comenzar con los ejercicios creare un odoo completamente nuevo en la version 18, realizando mas ajustes para no tener problemas constantes de nuevo.
\begin{verbatim}
 docker-compose up -d
 \end{verbatim}
    \begin{figure}[H]
    \centering
    \includegraphics[width=4in]{img1.png}
\end{figure}
Credenciales dentro de Odoo:
    \begin{figure}[H]
    \centering
    \includegraphics[width=4in]{img2.png}
\end{figure}
cr5h-z888-pktr
    \begin{figure}[H]
    \centering
    \includegraphics[width=4in]{img3.png}
\end{figure}
Cada vez que tengo que probar los cambios realizados, debo usar el comando en docker: 
\begin{verbatim}
 docker-compose restart odoo
 docker-compose up -d
 \end{verbatim}
Cuando tengo algun error total de odoo uso este comando, el cual borra todos los volumenes y 
    me permite crearlos de nuevo cuando se corrompen, cosa que es común al realizar cambios 
    relacionados con manejo de datos.
\begin{verbatim}
docker volume prune -f
 \end{verbatim}

\section{Actividad 01}
    \subsection{Objetivo}
Para realizar este ejercicio debo estudiar el modulo LigaFutbol y luego incluir las funciones de reglas, botones para goles de partidos, 
webcontroller para eliminar empates, crear inforemes PDF, Wizard para nuevos partidos y vista Graph.\\
    \subsection{Implementación}
     Usando de referencia base el archivo de EJ07 LigaFutbol, la estudiaremos y realizaremos los cambios necesarios.\\
     Como es normal Odoo no funciona correctamente debido al tree en vez de list, por lo que modifico todos los tree de las views.\\
     Tras esto analizo el primer punto.
     \subsection{01.1}
Primero necesito modificar la puntuación de partidos.\\
\begin{lstlisting}
        puntos = fields.Integer(default=0)
    #puntos= fields.Integer( compute="_compute_puntos",default=0, store=True)
\end{lstlisting}
Modifico el metodo para sumar 4 putnos al ganador y -1 al perdedor.
\begin{lstlisting}
    def actualizar_clasificacion(self):
        """
        Funcion para actualizar la clasificacion de todos los equipos
        """
#reinicio
        equipos = self.env['liga.equipo'].search([])
        for equipo in equipos:
            equipo.write({
                'victorias': 0,
                'empates': 0,
                'derrotas': 0,
                'goles_a_favor': 0,
                'goles_en_contra': 0,
                'puntos': 0,
            })
        
        partidos = self.env['liga.partido'].search([])
        for partido in partidos:
            goles_casa = partido.goles_casa or 0
            goles_fuera = partido.goles_fuera or 0
            diferencia = abs(goles_casa - goles_fuera)
            
 #Actualizar equipo local
            if partido.equipo_casa:
                equipo_casa = partido.equipo_casa
                nuevo_puntos_casa = equipo_casa.puntos
                nueva_victorias_casa = equipo_casa.victorias
                nueva_empates_casa = equipo_casa.empates
                nueva_derrotas_casa = equipo_casa.derrotas
                nuevo_goles_favor_casa = equipo_casa.goles_a_favor + goles_casa
                nuevo_goles_contra_casa = equipo_casa.goles_en_contra + goles_fuera
                
                if goles_casa > goles_fuera:
                    nueva_victorias_casa += 1
                    if diferencia >= 4:
                        nuevo_puntos_casa += 4  
                    else:
                        nuevo_puntos_casa += 3 
                elif goles_casa < goles_fuera:
                    nueva_derrotas_casa += 1
                    if diferencia >= 4:
                        nuevo_puntos_casa -= 1 
                else:
                    nueva_empates_casa += 1
                    nuevo_puntos_casa += 1  
                
                equipo_casa.write({
                    'victorias': nueva_victorias_casa,
                    'empates': nueva_empates_casa,
                    'derrotas': nueva_derrotas_casa,
                    'goles_a_favor': nuevo_goles_favor_casa,
                    'goles_en_contra': nuevo_goles_contra_casa,
                    'puntos': nuevo_puntos_casa,
                })
            
#visitante
            if partido.equipo_fuera:
                equipo_fuera = partido.equipo_fuera
                nuevo_puntos_fuera = equipo_fuera.puntos
                nueva_victorias_fuera = equipo_fuera.victorias
                nueva_empates_fuera = equipo_fuera.empates
                nueva_derrotas_fuera = equipo_fuera.derrotas
                nuevo_goles_favor_fuera = equipo_fuera.goles_a_favor + goles_fuera
                nuevo_goles_contra_fuera = equipo_fuera.goles_en_contra + goles_casa
                
                if goles_fuera > goles_casa:
                    nueva_victorias_fuera += 1
                    if diferencia >= 4:
                        nuevo_puntos_fuera += 4 
                    else:
                        nuevo_puntos_fuera += 3 
                elif goles_fuera < goles_casa:
                    nueva_derrotas_fuera += 1
                    if diferencia >= 4:
                        nuevo_puntos_fuera -= 1  
                else:
                    nueva_empates_fuera += 1
                    nuevo_puntos_fuera += 1 
                
                equipo_fuera.write({
                    'victorias': nueva_victorias_fuera,
                    'empates': nueva_empates_fuera,
                    'derrotas': nueva_derrotas_fuera,
                    'goles_a_favor': nuevo_goles_favor_fuera,
                    'goles_en_contra': nuevo_goles_contra_fuera,
                    'puntos': nuevo_puntos_fuera,
                })
        
        return True
\end{lstlisting}
Ahora probare las modificaciones. Creo un equipo.
    \begin{figure}[H]
    \centering
    \includegraphics[width=4in]{img6.png}
\end{figure}
Equipo de prueba creado, utilizando todas sus opciones.
    \begin{figure}[H]
    \centering
    \includegraphics[width=4in]{img7.png}
\end{figure}
Aparece en equipos.
    \begin{figure}[H]
    \centering
    \includegraphics[width=4in]{img8.png}
\end{figure}
Creamos un partido de prueba
    \begin{figure}[H]
    \centering
    \includegraphics[width=4in]{img9.png}
\end{figure}
Tras los cambios realizados, el equipo obtiene los puntos correctamente.
    \begin{figure}[H]
    \centering
    \includegraphics[width=4in]{img10.png}
\end{figure}
Numerosos problemas con la actualizacion de los datos, principalmente por odoo, al no saber si se implementaron los cambios, 
si dieron error o si necesitaba añadir otro partido para que actualice los cambios anteriores.\\




     \subsection{01.2}
Ahora debo modificar los botones de partidos.\\
Por lo que deberia de modificar liga partido.xml para añadir botones.\\
Debo editar tambien la view para añadir el button.\\
Test de los botones en models.
\begin{lstlisting}
    def gollocal(self):
        for partido in self.search([]):
            partido.goles_casa += 2
        self.actualizoRegistrosEquipo()

    def golvisit(self):
        for partido in self.search([]):
            partido.goles_fuera += 2
        self.actualizoRegistrosEquipo()

\end{lstlisting}
Views:
\begin{lstlisting}
                    <header>
                    <button name="gollocal"
                            type="object"
                            string="+2 para local"
                            class="btn-primary"/>

                    <button name="golvisit"
                            type="object"
                            string="+2 para visitante"
                            class="btn-secondary"/>
                </header>
\end{lstlisting}
Tras probar los botones, funcionan correctamente.
    \begin{figure}[H]
    \centering
    \includegraphics[width=4in]{img11.png}
    \end{figure}

     \subsection{01.3}
Debo eliminar empates.\\
Necesito crear un webcontroller nuevo por lo que creo este nuevo py.\\
\begin{lstlisting}
from odoo import http
from odoo.http import request

class LigaController(http.Controller):
#link correcto  http://localhost:8069/eliminarempates
    @http.route('/eliminarempates', auth='public', type='http')
    def eliminar_empates(self):
            Partido = request.env['liga.partido'].sudo()
            partidos = Partido.search([])
            empates = partidos.filtered(lambda p: p.goles_casa == p.goles_fuera)
            total = len(empates)
            empates.unlink()
            return f"Partidos eliminados: {total}"

\end{lstlisting}
Ahora debo añadirlo al init.py (sin guion bajo en latex, si no explota)\\
El resultado es correcto al acceder a la url.
    \begin{figure}[H]
    \centering
    \includegraphics[width=4in]{img12.png}
    \end{figure}
Me encontré con multiples errores 404 y 500 al intentar acceder a la url debido a problemas de odoo.\\



     \subsection{01.4}
Crear informe en pdf.\\
Como podemos ver en la carpeta report, el xml ya cuenta con parte de los elementos para crear un informe en pdf.\\
\begin{lstlisting}
    <!-- Realmente, este es el informe, lo de arriba es la plantilla que utilizara el informe -->
    <report id="report_clasificacion" model="liga.equipo" string="Informe clasificacion de cada equipo" name="EJ07-LigaFutbol.report_clasificacion_view" file="EJ07-LigaFutbol.report_clasificacion_view" report_type="qweb-pdf" />
\end{lstlisting}
Odoo deberia de crear automaticamente el informe al tener la plantilla y el informe definidos, pero como suele pasar
Odoo no lo genera correctamente, al no mostrar la opcion de imprimir que deberia de aparecer de base\\
El py:
\begin{lstlisting}
    class LigaPartidoWizard(models.TransientModel):
    _name = 'liga.partido.wizard'

    equipo_casa = fields.Many2one('liga.equipo', required=True)
    equipo_fuera = fields.Many2one('liga.equipo', required=True)
    goles_casa = fields.Integer()
    goles_fuera = fields.Integer()

    def crear_partido(self):
        self.env['liga.partido'].create({
            'equipo_casa': self.equipo_casa.id,
            'equipo_fuera': self.equipo_fuera.id,
            'goles_casa': self.goles_casa,
            'goles_fuera': self.goles_fuera,
        })
\end{lstlisting}
El xml:
\begin{lstlisting}
    <form string="Nuevo Partido">
    <group>
        <field name="equipo_casa"/>
        <field name="goles_casa"/>
        <field name="equipo_fuera"/>
        <field name="goles_fuera"/>
    </group>
    <footer>
        <button string="Crear" type="object" name="crear_partido" class="btn-primary"/>
        <button string="Cancelar" special="cancel"/>
    </footer>
</form>
\end{lstlisting}



    \subsection{01.5}
Crear nuevos partidos mediante wizard.\\
Tenemos el docuemnto base wizard ligapartidowizard.py de referencia, ahora lo adaptamos a partidos.\\
    \begin{figure}[H]
    \centering
    \includegraphics[width=4in]{img13.png}
    \end{figure}
Cree un boton debido a que odoo no queria mostrar el wizard en la vista de partidos automaticamente.\\
Este te permite crear nuevos partidos seleccionando ambos.
    \begin{figure}[H]
    \centering
    \includegraphics[width=4in]{img14.png}
    \end{figure}
    \subsection{01.6}
Por ultimo gráfica de estadisticas.\\
Esta consiste en generear una gráfica, ya contamos con un elemento de grafica en ligaequipo.xml, por lo que debemos adaptarlo 
sin mas.\\
\begin{lstlisting}
        <record model="ir.ui.view" id="liga_equipo_view_graph">
        <field name="name">Puntos por equipo</field>
        <field name="model">liga.equipo</field>
        <field name="type">graph</field>
        <field name="arch" type="xml">
             Indicamos que cada fila es un equipo y que la medida que los valora son los puntos
                Con esto conseguimos, por ejemplo, en un pie chart, que cada quesito sea un equipo
              <graph string="Puntos por equipo">
                <field name="nombre" group="True" type="row"/>
                <field name="puntos" group="True" type="measure"/>
             </graph>
         </field>
     </record>
\end{lstlisting}
En partidos:
\begin{lstlisting}
     <record id="view_liga_partido_graph" model="ir.ui.view">
        <field name="name">liga.partido.graph</field>
        <field name="model">liga.partido</field>
        <field name="arch" type="xml">
            <graph string="Goles de equipos locales" type="bar">
                <field name="equipo_casa" type="row"/>
                
                <field name="goles_casa" type="measure"/>
            </graph>
        </field>
    </record>
\end{lstlisting}
        
Tras completar el resto de ajustes tal y como la otra visa, podemos ver las graficas en partidos con los datos actuales.
    \begin{figure}[H]
    \centering
    \includegraphics[width=4in]{img15.png}
    \end{figure}




\section{Actividad 02}
    \subsection{Objetivo}
En este ejercicio realizaré un video explicando el funcionamiento del modulo api rest socios, teniendo que usar las opciones
crear, modificar,consulatar y eliminar.\\

Instalo Thunder CLient en Visual Studio, herramienta centrada en probar apis de forma sencilla

\begin{lstlisting}
    http://localhost:8069/
\end{lstlisting}
Crear: POST
\begin{lstlisting}
    http://localhost:8069/gestion/apirest/socio?data={"num_socio": 110, "nombre": "Pepe", "apellidos": "Suarez"}
\end{lstlisting}

Consultar: get
\begin{lstlisting}
    http://localhost:8069/gestion/apirest/socio?data={"num_socio": 110 }
\end{lstlisting}

Modificar: put
\begin{lstlisting}
    http://localhost:8069/gestion/apirest/socio?data={"num_socio": 110, "nombre": "Pepe", "apellidos": "Martinez"}
\end{lstlisting}
Consultar todo: get
\begin{lstlisting}
     http://localhost:8069/gestion/socio
\end{lstlisting}

Borrar: Delete
\begin{lstlisting}
    http://localhost:8069/gestion/apirest/socio?data={"num_socio": 110}
\end{lstlisting}


 Link al video:\\ \url{https://youtu.be/VpNurlFFFbw}




    








\section{Actividad 03}
    \subsection{Objetivo}
Crear un bot de telegram que realize ciertas ordenes mediante comandos.\\
    \subsection{Implementación}
Primero analizo el codigo, veo que  apirest es el controlador para los post,put,get y delete. \\
Listasocios.py se centra en devolver el json con los datos.\\
Socio.py es el modelo de datos.\\
Ahora debo crear el bot en telegram.\\
Documento telegramsociosbot. py:\\
\begin{lstlisting}
    import json
import re
import requests
from telegram import Update
from telegram.ext import Application, CommandHandler, MessageHandler, filters, ContextTypes

TOKEN_TELEGRAM = "8268402944:AAHcpcgy6tD9enSmWuZ2H7gMim-8Yxxe7f8" 
URL_API = "http://localhost:8069/gestion/apirest/socio"

def extraer_parametros(texto):
    resultado = {}
    patron = r'(\w+)="([^"]*)"'
    coincidencias = re.findall(patron, texto)
    
    for clave, valor in coincidencias:
        if clave == 'num_socio' and valor.isdigit():
            resultado[clave] = int(valor)
        else:
            resultado[clave] = valor
    
    return resultado

#/start
async def start(update: Update, context: ContextTypes.DEFAULT_TYPE):
    mensaje = (
        "COMANDOS DISPONIBLES:\n\n"
        "/crear nombre=\"nombre\",apellidos=\"apellidos\",num_socio=\"numero\"\n"
        "/modificar nombre=\"nombre\",apellidos=\"apellidos\",num_socio=\"numero\"\n"
        "/consultar num_socio=\"numero\"\n"
        "/borrar num_socio=\"numero\"\n\n"
        "EJEMPLOS:\n"
        "/crear nombre=\"Juan\",apellidos=\"Perez\",num_socio=\"101\"\n"
        "/consultar num_socio=\"101\""
    )
    await update.message.reply_text(mensaje)

#/crear
async def crear(update: Update, context: ContextTypes.DEFAULT_TYPE):
    if not context.args:
        await update.message.reply_text("Uso: /crear nombre=\"nombre\",apellidos=\"apellidos\",num_socio=\"numero\"")
        return
    
    texto = ' '.join(context.args)
    params = extraer_parametros(texto)
    
    if not all(k in params for k in ['nombre', 'apellidos', 'num_socio']):
        await update.message.reply_text("Faltan parametros: nombre, apellidos, num_socio")
        return

    datos = {
        "num_socio": params['num_socio'],
        "nombre": params['nombre'],
        "apellidos": params['apellidos']
    }
    
    try:
        respuesta = requests.post(URL_API, json=datos, headers={'Content-Type': 'application/json'})
        
        if respuesta.status_code == 200:
            socio = respuesta.json()[0]
            mensaje = f"Socio creado:\nNumero: {socio.get('num_socio')}\nNombre: {socio.get('nombre')}\nApellidos: {socio.get('apellidos')}"
        else:
            mensaje = f"Error: {respuesta.status_code}"
            
    except Exception as e:
        mensaje = f"Error de conexion: {str(e)}"
    
    await update.message.reply_text(mensaje)

#/modificar
async def modificar(update: Update, context: ContextTypes.DEFAULT_TYPE):
    if not context.args:
        await update.message.reply_text("Uso: /modificar nombre=\"nombre\",apellidos=\"apellidos\",num_socio=\"numero\"")
        return
    
    texto = ' '.join(context.args)
    params = extraer_parametros(texto)
    
    if 'num_socio' not in params:
        await update.message.reply_text("Falta parametro: num_socio")
        return

    datos = {
        "num_socio": params['num_socio'],
        "nombre": params.get('nombre', ''),
        "apellidos": params.get('apellidos', '')
    }
    
    try:
        respuesta = requests.put(URL_API, json=datos, headers={'Content-Type': 'application/json'})
        
        if respuesta.status_code == 200:
            socio = respuesta.json()[0]
            mensaje = f"Socio modificado:\nNumero: {socio.get('num_socio')}\nNombre: {socio.get('nombre')}\nApellidos: {socio.get('apellidos')}"
        elif respuesta.status_code == 404:
            mensaje = f"Socio {params['num_socio']} no encontrado"
        else:
            mensaje = f"Error: {respuesta.status_code}"
            
    except Exception as e:
        mensaje = f"Error de conexion: {str(e)}"
    
    await update.message.reply_text(mensaje)

#/consultar
async def consultar(update: Update, context: ContextTypes.DEFAULT_TYPE):
    if not context.args:
        await update.message.reply_text("Uso: /consultar num_socio=\"numero\"")
        return
    
    texto = ' '.join(context.args)
    params = extraer_parametros(texto)
    
    if 'num_socio' not in params:
        await update.message.reply_text("Falta parametro: num_socio")
        return

    try:
        parametros_url = {'data': json.dumps({"num_socio": params['num_socio']})}
        respuesta = requests.get(URL_API, params=parametros_url)
        
        if respuesta.status_code == 200:
            socio = respuesta.json()
            mensaje = f"Socio encontrado:\nNumero: {socio.get('num_socio')}\nNombre: {socio.get('nombre')}\nApellidos: {socio.get('apellidos')}"
        elif respuesta.status_code == 404:
            mensaje = f"Socio {params['num_socio']} no encontrado"
        else:
            mensaje = f"Error: {respuesta.status_code}"
            
    except Exception as e:
        mensaje = f"Error de conexion: {str(e)}"
    
    await update.message.reply_text(mensaje)

#/borrar
async def borrar(update: Update, context: ContextTypes.DEFAULT_TYPE):
    if not context.args:
        await update.message.reply_text("Uso: /borrar num_socio=\"numero\"")
        return
    
    texto = ' '.join(context.args)
    params = extraer_parametros(texto)
    
    if 'num_socio' not in params:
        await update.message.reply_text("Falta parametro: num_socio")
        return

    try:
        parametros_url = {'data': json.dumps({"num_socio": params['num_socio']})}
        respuesta = requests.delete(URL_API, params=parametros_url)
        
        if respuesta.status_code == 200:
            socio = respuesta.json()[0]
            mensaje = f"Socio eliminado:\nNumero: {socio.get('num_socio')}\nNombre: {socio.get('nombre')}\nApellidos: {socio.get('apellidos')}"
        elif respuesta.status_code == 404:
            mensaje = f"Socio {params['num_socio']} no encontrado"
        else:
            mensaje = f"Error: {respuesta.status_code}"
            
    except Exception as e:
        mensaje = f"Error de conexion: {str(e)}"
    
    await update.message.reply_text(mensaje)

async def mensaje_no_soportado(update: Update, context: ContextTypes.DEFAULT_TYPE):
    if update.message.text.startswith('/'):
        await update.message.reply_text("Orden no soportada")
    else:
        await update.message.reply_text("Envia un comando. Usa /start para ayuda.")

def main():
    print("Iniciando bot...")
    
    app = Application.builder().token(TOKEN_TELEGRAM).build()
    app.add_handler(CommandHandler("start", start))
    app.add_handler(CommandHandler("crear", crear))
    app.add_handler(CommandHandler("modificar", modificar))
    app.add_handler(CommandHandler("consultar", consultar))
    app.add_handler(CommandHandler("borrar", borrar))

    app.add_handler(MessageHandler(filters.TEXT & ~filters.COMMAND, mensaje_no_soportado))
    
    print("Bot listo. Presiona Ctrl+C para detener.")
    app.run_polling(allowed_updates=Update.ALL_TYPES)

if __name__ == '__main__':
    main()
\end{lstlisting}


Busco el botfather en telegram y creo un nuevo bot.
    \begin{figure}[H]
    \centering
    \includegraphics[width=4in]{img16.png}
    \end{figure}
    Comienzo y creo un nuevo bot.
        \begin{figure}[H]
    \centering
    \includegraphics[width=3in]{img17.png}
    \end{figure}
Tras configurarlo, ya esta creado correctamente.\\
Ahora debo obtener el token para usarlo en el codigo.
            \begin{figure}[H]
    \centering
    \includegraphics[width=3in]{img18.png}
    \end{figure}
Tras iniciar el bot en telegram y obtener el token, ejecuto el comando para isntalar las dependencias.\\
    \begin{figure}[H]
    \centering
    \includegraphics[width=4in]{img19.png}
    \end{figure}
Ahora creo el neuvo archivo para el bot. fundamental añadir el token.\\
\begin{lstlisting}
    import json
import re
import requests
from telegram import Update
from telegram.ext import Application, CommandHandler, MessageHandler, filters, ContextTypes

TOKEN_TELEGRAM = "8268402944:AAHcpcgy6tD9enSmWuZ2H7gMim-8Yxxe7f8" 
URL_API = "http://localhost:8069/gestion/apirest/socio"
\end{lstlisting}
Istalar en odoo:
    \begin{figure}[H]
    \centering
    \includegraphics[width=4in]{img21.png}
    \end{figure}
Error al añadir el bot, version no adecuada.\\
\begin{lstlisting}
    python -m pip uninstall python-telegram-bot
    python -m pip install python-telegram-bot
    python telegram_socios_bot.py
\end{lstlisting}
Tras esto:
    \begin{figure}[H]
    \centering
    \includegraphics[width=4in]{img20.png}
    \end{figure}
Ahora tras numerosos cambios:
    \begin{figure}[H]
    \centering
    \includegraphics[width=3in]{img22.png}
    \end{figure}
        \begin{figure}[H]
    \centering
    \includegraphics[width=3in]{img23.png}
    \end{figure}
        \begin{figure}[H]
    \centering
    \includegraphics[width=3in]{img24.png}
    \end{figure}
    Ya funciona correctamente permitiendo las funciones indicadas:
            \begin{figure}[H]
    \centering
    \includegraphics[width=3in]{img25.png}
    \end{figure}

\section{Actividad 04}
    \subsection{Objetivo}
Segun el ejemplo GenerarBarcode, crear un webcontroller que genere una imagen qr aleatoria.
    \subsection{Implementación}
Para ello necesito crear un nuevo controller usando los elementos del ejemplo dado, este genera el 
elemento random al añadirle los parametros a la url.\\

\begin{lstlisting}
    # -*- coding: utf-8 -*-
import base64
import io
import random
from PIL import Image
from odoo import http
from odoo.http import request, content_disposition

class RandomImageController(http.Controller):
    
    @http.route('/random_image', type='http', auth='public')
    def generate_random_image(self, width=100, height=100, **kwargs):
        """
        Genera una imagen con pixeles aleatorios  
        Parametros:
        - width: ancho de la imagen (por defecto 100)
        - height: alto de la imagen (por defecto 100)
        
        Ejemplos:
        - http://localhost:8069/random_image?width=300&height=200
        - http://localhost:8069/random_image?width=500 
        """
        try:
            width = int(width)
            height = int(height)
            if width <= 0 or height <= 0:
                return request.make_response(
                    'Error mas de 0.',
                    headers=[('Content-Type', 'text/plain')]
                )
            img = Image.new('RGB', (width, height), color='white')
            pixels = img.load()

            for i in range(width):
                for j in range(height):
                    red = random.randint(0, 255)
                    green = random.randint(0, 255)
                    blue = random.randint(0, 255)
                    pixels[i, j] = (red, green, blue)
            buffer = io.BytesIO()
            img.save(buffer, format='PNG')
            buffer.seek(0)
            
#PNG
            return request.make_response(
                buffer.getvalue(),
                headers=[
                    ('Content-Type', 'image/png'),
                    ('Content-Disposition', content_disposition('random_image.png'))
                ]
            )
            
        except ValueError:
            return request.make_response(
                headers=[('Content-Type', 'text/plain')]
            )
        except Exception as e:
            return request.make_response(
                f'Error generando imagen: {str(e)}',
                headers=[('Content-Type', 'text/plain')]
            )
    
    @http.route('/random_image_base64', type='http', auth='public')
    def generate_random_image_base64(self, width=100, height=100, **kwargs):
        """
        - width: ancho de la imagen (por defecto 100)
        - height: alto de la imagen (por defecto 100)
        
        Ejemplo:
        - http://localhost:8069/random_image_base64?width=300&height=200
        """
        try:
            width = int(width)
            height = int(height)
            
            if width <= 0 or height <= 0:
                return request.make_response(
                    headers=[('Content-Type', 'application/json')]
                )

            img = Image.new('RGB', (width, height), color='white')
            pixels = img.load()
            
            for i in range(width):
                for j in range(height):
                    red = random.randint(0, 255)
                    green = random.randint(0, 255)
                    blue = random.randint(0, 255)
                    pixels[i, j] = (red, green, blue)
 #64
            buffer = io.BytesIO()
            img.save(buffer, format='PNG')
            buffer.seek(0)
            img_base64 = base64.b64encode(buffer.getvalue()).decode('utf-8')
            
            return request.make_response(
                img_base64,
                headers=[('Content-Type', 'text/plain')]
            )
            
        except Exception as e:
            return request.make_response(
                f'{{"error": "{str(e)}"}}',
                headers=[('Content-Type', 'application/json')]
            )
\end{lstlisting}
Instalar pillow acordarse
\begin{lstlisting}
    python -m pip install python-barcode Pillow
\end{lstlisting}

Instalar el modulo
            \begin{figure}[H]
    \centering
    \includegraphics[width=4in]{img26.png}
    \end{figure}
Error hay que instalar los otros elementos del barcode antes:
            \begin{figure}[H]
    \centering
    \includegraphics[width=4in]{img27.png}
    \end{figure}
Eliminar el elemento antiguo de ejemplo arrelgo el problema, al ejercutarlo claramente estaba abriendo el elemento
antiguo de generador de codigo de barras en vez de la nueva aplicación
            \begin{figure}[H]
    \centering
    \includegraphics[width=4in]{img28.png}
    \end{figure}
Usando la url ahora podemos crear una imagen aleatoria:
\begin{lstlisting}
    http://localhost:8069/random_image?width=400&height=300
\end{lstlisting}

            \begin{figure}[H]
    \centering
    \includegraphics[width=4in]{img29.png}
    \end{figure}

\section{Conclusión}
Mi mayor problema al realizar estos ejercicios son los conflictos de versiones, teniendo problemas icluso 
lanzando alugnos de los modulos predeterminados.\\
Actualizar todo a la ultima version tambien me resulto inestable teniendo que usar versiones especificas
que no tienen problemas con odoo 18.\\
Me encontre con numerosos 404 pero por suerte esta ver fueron mas intuitivos, pudiendo arreglarlos de forma 
mas sencilla.

\end{document}